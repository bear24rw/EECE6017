\documentclass[11pt]{article}
\usepackage[margin=0.5in]{geometry}
\usepackage{titling}
\usepackage[compact]{titlesec}

\setlength{\droptitle}{-4em}
\addtolength{\droptitle}{-4pt}

\title{Embedded System Design \\ Buffer Overflow Tutorial}
\author{ Max Thrun }

\begin{document}
\maketitle

\section*{Problem 1}
\begin{itemize}
    \item \textbf{
            In the tutorial we have seen a popular implementation of a 
            'stack smashing attack' called 'code injection attack' which 
            made use of the NOP Sled technique to execute code on the stack 
            in order to make the processor call/ jump to 'critical\_function'. 
            Can the NOP Sled technique be applied on the hypothetical 
            application shown in Figure 1? Why or why not?
        }\\\\
        The application shown in Figure 1 is susceptible to the NOP sled
        technique. The array 'name' is only allocated to be 4 bytes but the
        strcpy function does not take this into account. It will keep writing
        bytes, starting at name[0], into memory until it reaches a null byte.
        The attacker could pass a string that is long enough to overwrite the
        return address on the stack, replacing it with a location somewhere
        in the NOPs. The program execution would then execute all the NOPs
        down to where the actual exploit code begins.
    \item \textbf{
            How would you make the processor execute (possibly in an 
            infinite loop) 'critical\_function' by buffer overflowing this 
            application? 
        }\\\\
        Like stated above, we would pass an input string to 'process\_input'
        large enough to overwrite the return address. We would have the
        execution jump to somewhere in the beginning of our buffer which
        will be filled with NOPs. The execution will continue down the NOP
        sled until it reaches our actual exploit code. The exploit code will
        simply jump to 'critical\_function' at 0x008002EC. If you want to 
        continue calling critical function you can add a jump at the end of 
        your exploit code to jump back a few instructions and call the 
        critical function again. We can modify the exploit string given in the
        tutorial to jump to 'critical\_function' at 0x008002EC:
        \begin{verbatim}
\x3a\xf0\x4a\x29\x3a\xf0\x4a\x29\x3a\xf0\x4a\x29\x3a\xf0\x4a\x29\x3a\xf0\x4a\x29\x3a
\xf0\x4a\x29\x3a\xf0\x4a\x29\x3a\xf0\x4a\x29\x3a\xf0\x4a\x29\x3a\xf0\x4a\x29\x3a\xf0
\x4a\x29\x3a\xf0\x4a\x29\x3a\xf0\x4a\x29\x3a\xf0\x4a\x29\x3a\xf0\x4a\x29\x3a\xf0\x4a
\x29\x3a\xf0\x4a\x29\x3a\xf0\x4a\x29\x3a\xf0\x4a\x29\x3a\xf0\x4a\x29\x3a\xf0\x4a\x29
\x34\x20\x40\x01\x04\x83\x40\x29\x3a\xe8\x3e\x28\x10\xab\x81\x00
        \end{verbatim}
\end{itemize}

\newpage
\section*{Problem 2}
\begin{itemize}
    \item \textbf{
            What (minimum) changes would you suggest in  the hypothetical 
            application shown in Figure 2 to be secure and prevent buffer 
            overflows? Assume that the arguments to the program (EmployeeName 
            and  EmployeeID) can be at max of MAX\_INPUT\_SIZE length, 
            including the null byte.
        }\\\\
        One simple solution would be to use the 'strlen()' function to ensure
        the length of EmployeeName and EmployeeID are < MAX\_INPUT\_SIZE. The new
        function could look like:
        \begin{verbatim}
void PrintRecord(char *EmployeeName, char *EmployeeID)
{
    char local_record[128];
    
    if (strlen(EmployeeName) > MAX_INPUT_SIZE ||
        strlen(EmployeeID) > MAX_INPUT_SIZE )
    {
        printf("Employee name and ID must be less than %d chars\n", MAX_INPUT_SIZE);
        return;
    }
    strcpy(local_record, EmployeeName);
    strcat(local_record, ":");
    strcat(local_record, EmployeeID);
    printf("The new record to be saved is %s", local_record);

    return;
}
\end{verbatim}
\end{itemize}

\section*{Problem 3}
\begin{itemize}
    \item \textbf{
            'Code injection attack' executes code using NOP Sled technique 
            mostly to spawn a shell by injecting code, for example with the 
            help of network packets. Explain how the Intrusion Detection 
            System (IDS) can be employed to prevent spawning a new shell. 
            Also explain how an attacker can try to avoid getting caught 
            by IDS.
        }\\\\
        An Intrusion Detection System could monitor system calls and look
        for suspicious calls such as exec(), which may be used to spawn
        a shell. It could then kill the child process. If the IDS is
        implemented at the kernel level it could even just prevent the
        child process from spawning to begin with. As an attacker, a
        possible way around this would be to not spawn a shell already
        on the system but to create and send your own as part of the exploit
        payload. Upon executing the payload the process essentially 
        \emph{becomes} shell with no need to launch an external one. A super
        advanced exploit may even execute the shell code in a separate thread
        to allow the original application to continue running.
\end{itemize}

\newpage
\section*{Problem 4}
\begin{itemize}
    \item \textbf{
            What a stack overflow is and its relevance in embedded system design.
        }\\\\
        A stack overflow is when the execution stack grows beyond the fixed memory
        that is reserved for it. For example if you have a recursive function that
        repeatedly adds return addresses on to the stack eventually you will run
        out of space and your stack will overflow. This is relevant to embedded
        systems because most embedded systems will be working with a fixed stack
        size. This requires the programmers to be careful when dealing with deep
        function calls.
    \item \textbf{
            How it is different from stack buffer overflow.
        }\\\\
        A buffer overflow is any case where a program writes beyond the memory
        allocated to it. For example if you allocate an array with length 4 but
        then strcpy a string a length 5 into it you've caused a buffer overflow.
        As shown in the above questions, a buffer overflow can be used to corrupt
        the stack and enable an attacker to run arbitrary code.
\end{itemize}

\section*{Problem 5}
\begin{itemize}
    \item \textbf{
            Explain various steps involved in the embedded system design process.
            Explain what security  measures you would incorporate in those 
            design steps to design a secure system.
        }\\\\
        The embedded system design process can be broken down into the following
        \begin{enumerate}
            \item Requirements
            \item Specification
            \item System architecture development
            \item Integration and test
        \end{enumerate}
        Of these steps the most important to pay careful attention to security
        would be system development and testing. Most exploits are a result of
        simple programming errors, such as not checking the length of the string
        before doing a strcpy. It is unlikely that theses types of issues would
        be brough up in any kinda of customer requirements or specification 
        discussion and so it is purely up to the developer to implement good
        coding techniques. Testing is equally as important to security and 
        testing should not only cover the 'working' conditions such as 
        entering names like 'Jane' and 'Bob' but also extreme conditions 
        such as names like 'AAAAAAAAAAAAAAAAAA'. Part of the testing strategy 
        should be to break the program in an attempted to tease out corner 
        cases and unaccounted conditions.
\end{itemize}

\section*{Problem 6}
\begin{itemize}
    \item \textbf{
             Explain the Address Space Layout Randomization (ASLR) security 
             technique and its use. Do you think it's a 100\% foolproof 
             method to prevent buffer overflow attacks? Why or why not?
        }\\\\
        The address space layout randomization technique involves randomly
        arranging the positions of things such as the base address of
        executable, the stack, and loaded libraries, in the processes
        address space. This makes it difficult to predict the target address
        for certain attacks. For example if you are trying to execute
        shell code which you injected on the stack you must first \emph{find}
        the stack. The ASLR technique relies on the hope that the attacker
        will not be able to guess or calculate the position of key structures.
        I do not think this method is 100\% foolproof because there is still
        a chance that the attacker could guess the location. There might also
        be a way to calculate and predict the locations if you have access to
        the ASLR algorithm.
\end{itemize}



\end{document}
