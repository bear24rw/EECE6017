\documentclass[12pt]{article}
\usepackage[margin=0.5in]{geometry}
\usepackage{titling}
\usepackage[compact]{titlesec}

\setlength{\droptitle}{-4em}
\addtolength{\droptitle}{-4pt}

\title{Embedded System Design \\ Readers Writers}
\author{ Max Thrun | Ian Cathey | Mark Labbato }

\begin{document}
\maketitle


\section*{Description}

The purpose of this lab was to create a system where one task, 
called the "writer", continuously writes words into a circular buffer while
multiple "reader" tasks read all the words that have been written. The
reader tasks are to execute simultaneously each each other but exclusive
from the writer task. To do this the system uses a combination of mutexes
to control access to the shared buffered. The first issue is that we need
to keep track of how many readers are currently accessing the buffer. To
do this we use a variable called "num\_readers" and a mutex called
"mutex\_num\_readers". Each reader pends on access to this mutex and once
it gains access it increments the "num\_readers". When the reader is done
accessing the shared buffer it tries to grain the mutex and then decrements
"num\_readers" since it is now finished. In order to lock out the writer
task each reader checks the value of "num\_readers" when it first gains access
to it. If it sees that it is the first reader (num\_readers == 0) it pends on
"mutex\_wr" which is a mutex that gets locked by the writer task when it is
writing. In order to ensure that the writer is not starved we use a third
mutex called "mutex\_rd". The writer waits on this mutex which causes it
execute as soon as possible since it has the highest priority. The reader
tasks lock the "mutex\_rd" during the critical section of keeping track of 
how many readers are running but it releases it immediately after. The delays
in the readers are varied by their priority to ensure we are able to observe
proper operation.


\end{document}
