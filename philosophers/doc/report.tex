\documentclass[12pt]{article}
\usepackage[margin=0.5in]{geometry}
\usepackage{titling}
\usepackage[compact]{titlesec}
\usepackage{url}

\setlength{\droptitle}{-4em}
\addtolength{\droptitle}{-4pt}

\title{Embedded System Design \\ Dining Philosophers}
\author{ Max Thrun | Ian Cathey | Mark Labbato }

\begin{document}
\maketitle

\section*{Description}

The purpose of this lab was to implement a solution to the
'Dining Philosophers Problem'. In short, the problem describes
5 philosophers all sitting around a circular table. Each philosopher
shares a fork with the person next to him. A philosopher can either
be thinking, in which he holds no forks, or eating, in which he holds
both. If a fork is not available then he waits until it
does become available. To solve this problem we implemented the common
resource hierarchy solution \cite{sol}. Our solution is based on the
convention that each fork is numbered and the philosopher will pick up
the fork with the lowest number first. This breaks the deadlock of each
person simply picking up the fork to their left. For instance, with 5
philosophers philosopher 5 and philosopher 1 both share fork 1. Fork 1
is the lowest value fork for both these philosophers so they cannot both
pick it up at the same time. This fact prevents the deadlock of all the
philosophers taking one fork from a particular side.
\\\\
From our programs perspective, when each thread starts it calculates the
value of the fork on its left and right. Then, after a random 'thinking'
delay, it checks to see which fork has the lower value. If the fork on the
left has the lowest value it tries to pick that one up first and vica versa.
Each fork is represented by a mutex which ensures that it can only be touched
by one thread (aka philosopher) at a time. After the philosopher has obtained both
forks it delays for a random amount of time while it 'eats' and then releases
both forks.
\\\\
Additionally, a pseudo-random number generator, based on a LFSR, was developed
and fitted with a mutex lock to make it thread safe. Using random numbers for our delays 
allowed us to avoid time aliasing which might occur with fixed length delays. Time aliasing 
can cause conditions such as deadlocks and starvation to remain hidden since the threads are
executing the exact same sequence and timing over and over. If that specific time interval
happens to be deadlock free you will never see it occur.  While the LFSR does not provide 
truly random delays it allows us to have more confidence in the robustness of our solution.

\begin{thebibliography}{9}
    \bibitem{sol} \url{http://en.wikipedia.org/wiki/Dining_philosophers_problem#Resource_hierarchy_solution}
\end{thebibliography}

\end{document}
